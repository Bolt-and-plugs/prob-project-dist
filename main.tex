\documentclass[a4paper,times,12pt]{article}
\usepackage{amsthm}
\usepackage[figuresright]{rotating}
\usepackage{graphicx}
\usepackage{booktabs}
\graphicspath{ {./images/} }
\usepackage{amssymb}
\usepackage{graphicx}
\usepackage{fancybox}
\usepackage{amsmath}
\usepackage{picinpar}
\usepackage{colortbl}
\usepackage{wasysym}
\usepackage{txfonts}
\usepackage{pb-diagram}
\usepackage{relsize}
\usepackage{tikz}
\usepackage{pgfplots}
\usepackage{subfigure}
\usepackage{algorithm}
\usepackage{algorithmic}
\usepackage{geometry}

\geometry{
  a4paper,        
  left=2.5cm,     
  right=2.5cm,    
  top=2.5cm,      
  bottom=2.5cm    
}

\begin{document}
\title{ANÁLISE DE DIFERENTES ALGORITMOS DE APRENDIZADO DE MÁQUINA PARA RECOMENDAÇÃO DE RESULTADOS NA FRANQUIA COUNTER STRIKE}
\author{Carlos Eduardo Nogueira Silva\\Felipe Gomes da Silva \\Felipe Matheus Possari\\Renan Sinhorini Pimentel\\ \\Instituto de Bioci\^{e}ncias, Letras e Ci\^{e}ncias Exatas, Unesp - \\ Univ Estadual Paulista (S\~{a}o Paulo State University) Rua Crist\'{o}v\~{a}o \\ Colombo 2265, Jd Nazareth, 15054-000, S\~{a}o Jos\'{e} do Rio Preto - SP, \\ Brazil.}
\maketitle
%%%%%%%%%%%%%%%%%%%%%%%%%%%%%%%%%%%%%%%%%%%%%%%%%%%%%%%%%%%%%%%%%%%%%%%%%%%%%
%%%%%%%%%%%%%%%%%%%%%%%%%%%%%%%%%%%%%%%%%%%%%%%%%%%%%%%%%%%%%%%%%%%%%%%%%%%%%

\section{Introdução}
\hspace{+15pt} 

\section{Objetivos}
\hspace{+15pt}

\section{Justificativa}
\hspace{+15pt}

\section{Coleta de dados}
\hspace{+15pt}

\section{5W1H}
\subsection{What – O que vamos fazer?}
\hspace{+15pt}
Faremos um estudo da base de dados do Kaggle: CS:GO Professional Matches, que contém mais de três mil partidas profissionais de Counter Strike. 
O Projeto busca realizar previsões de resultados de partidas futuras com base na a análise dos dados dessa base, além de fornecer dados importantes sobre os times como suas forças e fraquezas em diferentes situações.

\subsection{Where – Para onde estamos olhando?}
\hspace{+15pt}
Utilizaremos dados de times do mais alto nível de competição do jogo, os 30 melhores times, ranqueados pelo site hltv.org, uma plataforma muito conhecida no meio competitivo de CSGO, dos maiores campeonatos dos anos em questão. 
O estudo terá ênfase nas estatísticas gerais do time, sem muito foco em dados individuais de cada jogador, uma vez que esses dados não refletem corretamente o desempenho da equipe como um todo.

\subsection{When – Para quando estamos olhando?}
\hspace{+15pt}
A base escolhida possui dados de partidas jogadas em grandes campeonatos entre os anos 2016 e 2020, portanto o estudo busca prever os resultados das partidas dos anos posteriores a 2020.

\subsection{Who – Para quem é este estudo?}
\hspace{+15pt}
O público alvo desse estudo são usuários de plataformas de aposta, em específico as que possuem a categoria Counter Strike, uma vez que com os resultados que buscamos, será possível prever ganhadores de futuras partidas. Dessa forma esperamos poder guiar os apostadores a maiores lucros.
Mesmo não sendo o público principal, também buscamos auxiliar jogadores de times profissionais e suas respectivas comissões técnicas, uma vez que não analisaremos apenas resultados e sim dados gerais dos times.

\subsection{Why – Por que estamos fazendo este estudo?}
\hspace{+15pt}
O motivo da escolha do tópico se dá não só pela afinidade que possuímos com o jogo, mas principalmente da escassez de estudos estatísticos e probabilísticos aprofundados nessa área. 
Com o aumento da popularidade não somente do jogo mas também das plataformas de aposta, possuímos um grande incentivo e um público cada vez maior capaz de usufruir dos resultados e análises do nosso estudo.

\subsection{How - Como vamos fazer?}
\hspace{+15pt}



%%%%%%%%%%%%%%%%%%%%%%%%%%%%%%%%%%%%%%%%%%%%%%%%%%%%%%%%%%%%%%%%%%%%%%%%%%%%%
%%%%%%%%%%%%%%%%%%%%%%%%%%%%%%%%%%%%%%%%%%%%%%%%%%%%%%%%%%%%%%%%%%%%%%%%%%%%%
\newpage
\begin{thebibliography}{00}

\bibitem{Predicting cs outcome} CHALMERS. Predicting the outcome of CS games using machine learning. 2018. Disponível em: https://publications.lib.chalmers.se/records/fulltext/256129/256129.pdf. Acesso em: [01 out. 2024].

\bibitem{Predicting cs matches} LUND UNIVERSITY. Predicting Counter-Strike Matches. 2024. Disponível em: https://lup.lub.lu.se/luur/download?func=downloadFile\&recordOId=9145457. Acesso em: [02 out. 2024].

\bibitem{HLTV} HLTV.org: your source for esports information. Disponível em: https://www.hltv.org. Acesso em: [29 set. 2024].

\bibitem{cs-2} SVEC, O. Predicting Counter-Strike Game Outcomes with Machine Learning. Disponível em: https://dspace.cvut.cz/bitstream/handle/10467/99181/F3-BP-2022-Svec-Ondrej-predicting\_csgo\_outcomes\_with\_machine\_learning.pdf. Acesso em: [02 out. 2024].

\end{thebibliography}
\end{document}
